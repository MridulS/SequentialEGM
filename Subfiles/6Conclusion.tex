\input{./econtexRoot.texinput}
\documentclass[\econtexRoot/SequentialEGM]{subfiles}
\onlyinsubfile{\externaldocument{\econtexRoot/SequentialEGM}}
\usepackage{\econtexSetup,\econark,\econtexShortcuts}

\begin{document}

\hypertarget{conclusion}{}
\par\section{Conclusion}
\notinsubfile{\label{sec:conclusion}}

% Summarize the method: Begin your conclusion by summarizing the new computational method you developed or proposed. Provide a brief overview of the key features of the method and how it differs from existing methods.

This paper introduces a novel method for solving dynamic stochastic optimization problems called the Sequential Endogenous Grid Method (EGM$^n$). Given a problem with multiple decision choices (or control variables), the Sequential Endogenous Grid Method proposes separating the problem into a sequence of smaller subproblems that can be solved sequentially by using more than one EGM step. Then, depending on the resulting endogenous grid from each subproblem, this paper proposes different methods for interpolating functions on non-rectilinear grids, called the Warped Homotopic Interpolation Method (WHIM) and the Gaussian Process Regression (GPR) method.

EGM$^n$ is similar to the Nested Endogenous Grid Method (NEGM)\footnote{\cite{Druedahl2021-wl}.} and the Generalized Endogenous Grid Method (G2EGM)\footnote{\cite{Druedahl2017-vn}.} in that it can solve problems with multiple decision choices, but it differs from these methods in that by choosing the subproblems strategically, we can take advantage of multiple sequential endogenous grid method steps to solve complex multidimensional models in a fast and efficient manner. Additionally, the use of machine learning tools such as the GPR overcomes bottlenecks seen in unstructured interpolation using Delauany triangulation and other similar methods.

% Evaluate the method: Evaluate the strengths and limitations of the new computational method you developed or proposed. Discuss how the method compares to existing methods in terms of accuracy, efficiency, and ease of use.

% Demonstrate the method: If possible, provide an example of how the new computational method can be used to solve a problem or answer a research question. This will help the reader understand the practical implications of the method.

% Highlight potential applications: Discuss potential applications of the new computational method. This will help demonstrate the broader impact of the method beyond the specific problem or research question addressed in your paper.

% Discuss future directions: Provide suggestions for future research based on the new computational method you developed or proposed. This can include improvements to the method, potential extensions to other areas of research, or new applications of the method.

% Conclude with final thoughts: End your conclusion with some final thoughts that tie together the main points of your paper. This will help leave a lasting impression on the reader.

\ifthenelse{\boolean{Web}}{}{
\end{document} \endinput
}
