%!TEX TS-program = xelatex
%!TEX encoding = UTF-8 Unicode
% Awesome CV LaTeX Template for Cover Letter
%
% This template has been downloaded from:
% https://github.com/posquit0/Awesome-CV
%
% Authors:
% Claud D. Park <posquit0.bj@gmail.com>
% Lars Richter <mail@ayeks.de>
%
% Template license:
% CC BY-SA 4.0 (https://creativecommons.org/licenses/by-sa/4.0/)
%

%-------------------------------------------------------------------------------
% CONFIGURATIONS
%-------------------------------------------------------------------------------
% A4 paper size by default, use 'letterpaper' for US letter
\documentclass[11pt, letterpaper]{awesome-cv}

% Configure page margins with geometry
\geometry{left=1.4cm, top=.8cm, right=1.4cm, bottom=1.8cm, footskip=.5cm}

% Specify the location of the included fonts
\fontdir[fonts/]

% Color for highlights
% Awesome Colors: awesome-emerald, awesome-skyblue, awesome-red, awesome-pink, awesome-orange
% awesome-nephritis, awesome-concrete, awesome-darknight
\colorlet{awesome}{awesome-concrete}
% Uncomment if you would like to specify your own color
% \definecolor{awesome}{HTML}{CA63A8}

% Colors for text
% Uncomment if you would like to specify your own color
% \definecolor{darktext}{HTML}{414141}
% \definecolor{text}{HTML}{333333}
% \definecolor{graytext}{HTML}{5D5D5D}
% \definecolor{lighttext}{HTML}{999999}

% Set false if you don't want to highlight section with awesome color
\setbool{acvSectionColorHighlight}{true}

% If you would like to change the social information separator from a pipe (|) to something else
\renewcommand{\acvHeaderSocialSep}{\quad\textbar\quad}

%-------------------------------------------------------------------------------
%	PERSONAL INFORMATION
%	Comment any of the lines below if they are not required
%-------------------------------------------------------------------------------
% Available options: circle|rectangle,edge/noedge,left/right
\photo[rectangle,edge,left]{profile_AL}
\name{Alan}{Lujan}
\position{Economics Ph.D. Candidate{\enskip\cdotp\enskip}The Ohio State University}
\address{5032 Baffin Bay Ln., Rockville, MD 20853}

\mobile{(832) 567 - 2665}
\email{alanlujan91@gmail.com}
\homepage{quantmacro.org}
\github{alanlujan91}
\linkedin{alanlujan91}
% \gitlab{gitlab-id}
% \stackoverflow{SO-id}{SO-name}
% \twitter{@twit}
% \skype{skype-id}
% \reddit{reddit-id}
% \medium{madium-id}
% \googlescholar{googlescholar-id}{name-to-display}
%% \firstname and \lastname will be used
% \googlescholar{googlescholar-id}{}
% \extrainfo{extra informations}

% \quote{``Be the change that you want to see in the world."}

%-------------------------------------------------------------------------------
%	LETTER INFORMATION
%	All of the below lines must be filled out
%-------------------------------------------------------------------------------
% The company being applied to
\recipient
 {Econometric Society Summer School in Dynamic Structural Econometrics}
 {Deep Learning for Solving and Estimating Dynamic Models \\ University of Lausanne, Switzerland}
% The date on the letter, default is the date of compilation
\letterdate{\today}
% The title of the letter
\lettertitle{Personal Statement}
% How the letter is opened
\letteropening{To whom it may concern,}
% How the letter is closed
\letterclosing{Best regards,}
% Any enclosures with the letter
\letterenclosure[Attached]{Curriculum Vitae}

%-------------------------------------------------------------------------------
\begin{document}

% Print the header with above personal informations
% Give optional argument to change alignment(C: center, L: left, R: right)
\makecvheader[R]

% Print the footer with 3 arguments(<left>, <center>, <right>)
% Leave any of these blank if they are not needed
\makecvfooter
 {\today}
 {Alan Lujan~~~·~~~Cover Letter}
 {Rockville, MD}

% Print the title with above letter informations
\makelettertitle

%-------------------------------------------------------------------------------
%	LETTER CONTENT
%-------------------------------------------------------------------------------
\begin{cvletter}

% \lettersection{About Me} Customize First Sentence

I am a Ph.D. candidate at The Ohio State University and work under the direction of Professor Christopher Carroll (Johns Hopkins University) whom I met when he recruited me to work on the \href{https://github.com/econ-ark/HARK}{\texttt{Econ-ARK/HARK}} project. I have been working with Chris and have been an \texttt{Econ-ARK/HARK} collaborator for 3 years . My areas of specialization are Quantitative Macroeconomics, Computational Economics, and Household Finance. I expect to complete my Ph.D. by December of 2023.

My primary research interest is understanding the financial decisions of households and exploring how macroeconomic policies can help mitigate inequality and precarity. To achieve this, I develop quantitative models that analyze a range of questions in areas including housing and mortgage choice, portfolio choice, and household spending on children's development. I am motivated by the potential impact of my research on understanding how conditions change over the business cycle for marginalized communities.

My job market paper, "EGM$^n$: The Sequential Endogenous Grid Method," proposes an extension the the Endogenous Grid Method (EGM) in a multidimensional setting. The method introduces a novel way of breaking down complex problems into a sequence of simpler, smaller, and more tractable problems that can use multiple EGM steps; examples in the paper show that each decomposition of multidimensional EGM problems into a sequence of lower-dimensional EGM stages can speed up the solution by orders of magnitude. Additionally, the paper highlights the value of modern machine learning techniques for multidimensional interpolation such as the Gaussian Process Regression advocated by Simon Schideger to tackle these problems effectively.

% why you would benefit from attending the DSE2021 summer school

I believe my research would greatly benefit from attending the DSE2023 summer school. My job market paper was inspired by Simon Schideger's work on Gaussian Process Regressions for solving high-dimensional problems. The work on discrete-continuous dynamic models by Iskhakov, Rust, and Schjerning has also been critical for solving complex models efficiently, and my research agenda includes extending their work to multidimensional sequential models with discrete choice. John Stachurski's work on abstract dynamic programming provides a robust framework for defining problems generally, and I am using these ideas to extend and generalize the \href{https://github.com/econ-ark/HARK}{\texttt{Econ-ARK/HARK}} toolkit.

Attending DSE2023 summer school would greatly benefit my research. My job market paper was inspired by Simon Schideger's work on Gaussian Process Regressions for solving high-dimensional problems. I also plan to extend Iskhakov, Rust, and Schjerning's work on discrete-continuous dynamic models to multidimensional sequential models with discrete choice. John Stachurski's abstract dynamic programming framework provides a robust problem definition, which I am using to generalize the \href{https://github.com/econ-ark/HARK}{\texttt{Econ-ARK/HARK}} toolkit.

Attending the DSE2023 summer school would be invaluable to my research. Specifically, I am interested in expanding upon Simon Schideger's work on Gaussian Process Regressions for solving high-dimensional problems, which has been a major inspiration for my job market paper. Additionally, I have found the work on discrete-continuous dynamic models by Iskhakov, Rust, and Schjerning to be critical for efficiently solving complex models, and I plan to extend their ideas to multidimensional sequential models with discrete choice. John Stachurski's work on abstract dynamic programming provides a robust framework for defining problems generally, and I believe that incorporating these ideas into the \href{https://github.com/econ-ark/HARK}{\texttt{Econ-ARK/HARK}} toolkit will greatly enhance its capabilities. Overall, attending the DSE2023 summer school would allow me to deepen my understanding of these topics and connect with other researchers in the field, ultimately advancing my research goals.


My hope is both that my work will benefit from what I learn and that it will be reciprocally interesting to the other students and possibly even the organizers.

\textbf{Reference}

\href{http://www.econ2.jhu.edu/people/ccarroll/index.html}{Christopher D. Carroll} \\
email: \href{mailto:ccarroll@jhu.edu}{ccarroll@jhu.edu}

\end{cvletter}

%-------------------------------------------------------------------------------
% Print the signature and enclosures with above letter informations
\makeletterclosing

\end{document}
