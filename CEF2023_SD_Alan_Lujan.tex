\input{./econtexRoot.texinput}\documentclass[SequentialEGM]{subfiles}
\externaldocument{\econtexRoot/SequentialEGM}
% \owner determines where links to online content go
% llorracc is Chris Carroll's personal version
% econ-ark is the Econ-ARK/REMARK version
% alanlujan91 is Alan Lujan's personal version
%\providecommand{\owner}{llorracc}
%\providecommand{\owner}{econ-ark}
\providecommand{\owner}{alanlujan91}

\usepackage{\econtexSetup,\econark,\econtexShortcuts}

\title{\Large EGM$^n$: The Sequential Endogenous Grid Method}
\author{Alan Lujan\authNum}
\renewcommand{\forcedate}{February 28, 2023}\date{\forcedate}



\begin{document}

\maketitle

\begin{authorsinfo}
    \noindent \name{\href{https://quantmacro.org}{Department of Economics, The Ohio State University}, \href{mailto:alanlujan91@gmail.com}{\texttt{alanlujan91@gmail.com}}.}
\end{authorsinfo}

\section{Paper's Contribution to Computational Economics}

This paper proposes a new extension to the Endogenous Grid Method (EGM) in a multidimensional setting. The method is called the Sequential Endogenous Grid Method (or EGM$^n$) and introduces a novel way of breaking down complex problems into a sequence of simpler, smaller, and more tractable problems that can use multiple EGM steps. Additionally, the paper illustrates modern machine learning methods for multidimensional interpolation that can be used to solve these problems.

The Sequential EGM consists of 3 major parts: First, I provide a guide to strategically and intuitively break up a model into a sequence of smaller problems that themselves don't add any additional state variables or introduce spurious dynamics. Second, I discuss the conditions that a subproblem must meet in order to use an EGM step and evaluate each of the smaller problems to see if they can be solved using an EGM step. If the subproblem cannot be solved with EGM, then convex optimization must be used. Third, if the exogenous grid generated by the EGM is non-rectangular, then I present two multidimensional interpolation methods that can be used with curvilinear grids or unstructured grids respectively. If the grid is curvilinear, I develop a Warped Grid Interpolator that can be run on a GPU to speed up the interpolation process. If the grid is unstructured, I use a Gaussian Process Regression (GPR) to generate an interpolating function and simultaneously assess the uncertainty of the prediction.

Solving each subproblem using this process, the EGM$^n$ is capable of handling complex problems that are not solvable with traditional EGM and are difficult and time-consuming to solve with convex optimization and grid search methods. In particular, the EGM$^n$ is different from G2EGM and NEGM in that it allows for using more than one EGM step to solve a problem, avoiding costly grid search methods to the extent that the problem allows. Additionally, the use of machine learning tools such as the GPR overcomes bottlenecks seen in unstructured interpolation using Delaunay triangulation and other similar methods.

The software for this project is being developed as part of the \href{https://github.com/econ-ark/HARK}{\texttt{Econ-ARK/HARK}} toolkit and can be found at \href{https://github.com/alanlujan91/HARK}{\texttt{alanlujan91/HARK}}, \href{https://github.com/alanlujan91/multinterp}{\texttt{alanlujan91/multinterp}},  and \href{https://gitfront.io/r/alanlujan91/DPFn3R7x3Avr/SequentialEGM/}{\texttt{alanlujan91/SequentialEGM}}.

\end{document}
